\documentclass{exam}
\usepackage{amsmath}

\pagestyle{headandfoot}
\firstpageheadrule
\runningheadrule
\firstpageheader{Prof. Pham \\ Mathematical Theory of Probability}{Homework 4}{Aadhithya Saravanan}
\runningheader{Mathematical Theory of Probability \\ Homework 4}{}{Saravanan}
\firstpagefooter{}{}{}
\runningfooter{ }{\thepage}{ }

\printanswers

\begin{document}

\underline{Problem 1}
\newline
\newline
Using conditional thinking, the size of the sample space is $6^2-6$, the number of ways to roll two dice and get different numbers. The number of ways to get at least one 6 is $5\cdot 2$. Thus, the probability of getting at least one 6 given that the numbers are different is $\frac{5\cdot 2}{6^2-6}$.
\newline

\underline{Problem 2}
\newline
\newline
Choosing 4 balls from $6+9$ balls gives the sample space a size of $\binom{15}{4}$. For the first two balls to be white, the first can be 1 of 6 balls and the second can be 1 of 5 balls. For the last two balls to be black, the first can be 1 of 9 balls and the second can be 1 of 8 balls. Thus, the probability of choosing 2 white and 2 black balls is $\frac{6\cdot 5\cdot 9\cdot 8}{\binom{15}{4}}$.
\newline

\underline{Problem 3}
\newline
\begin{parts}
    \part Using conditional thinking, the sample space of the possible positions of the 3 white balls is $S={(1,2,3),(1,2,4),(1,3,4),(2,3,4)}$ when the 4 balls are drawn with replacement. Two of these four outcomes have white balls in both the first and third positions. Thus, the probability of having white balls in both the first and third positions is $\frac{2}{4}$.
    \part Without replacement, the problem does not change because each outcome still has exactly 3 white balls and their positions are still equally likely. Thus, the probability of having white balls in both the first and third positions is still $\frac{2}{4}$.
    \newline
\end{parts}

\underline{Problem 4}
\newline
\newline
The sample space has size $6\cdot12\cdot4$. The number of ways to get exactly 2 white balls is $2\cdot8\cdot3+2\cdot4\cdot1+4\cdot8\cdot1$. The number of ways to get exactly 2 white balls and the ball from urn A to be white is $2\cdot8\cdot3+2\cdot4\cdot1$. Thus, the probability of getting a white ball from urn A given that there are exactly 2 white balls is $\frac{2\cdot8\cdot3+2\cdot4\cdot1}{6\cdot12\cdot4}/\frac{2\cdot8\cdot3+2\cdot4\cdot1+4\cdot8\cdot1}{6\cdot12\cdot4}=\frac{2\cdot8\cdot3+2\cdot4\cdot1}{2\cdot8\cdot3+2\cdot4\cdot1+4\cdot8\cdot1}$.
\newline

\underline{Problem 5}
\newline
\newline
Using conditional thinking, the choices for the first card given that the second and third are spades are the 11 remaining spades. There are 50 cards left in the deck after the second and third cards are drawn. Thus, the probability of the first card being a spade given that the second and third cards are spades is $\frac{11}{50}$.
\newline

\underline{Problem 6}
\newline
\newline
Using conditional thinking, there are a total of 3 ways to arrange the 3 cards so that A's value is less than B's value, so $F={(A,C,B), (A,B,C), (C,A,B)}$. Two of these arrangements have A before C, so the probability of A's value being less than C's value given that A's value is less than B's value is $\frac{2}{3}$.
\newline

\underline{Problem 7}
\newline
\begin{parts}
    \part The chance of the first 4 flips all being heads is $p^4$.
    \part The chance of the first flip being tails and the next 3 flips being heads is $(1-p)p^3$.
    \part The chance of getting HHHH before THHH can be found easily because the only way to get HHHH is if a tails never comes up. So, that means the first 4 flips must be heads, which has a probability of $p^4$.
    \newline
\end{parts}

\underline{Problem 8}
\newline
\begin{parts}
    \part The probability of both shots hitting given that at least one hit can be found using the formula $P(A|B)=\frac{P(A\cap B)}{P(B)}$. Let E be the event that Barbara hit and F be the event that Diane hit. We are looking for $P(E\cap F|E\cup F)$. Using the formula, we have $P(E\cap F|E\cup F)=\frac{P(E\cap F)\cap(E\cup F)}{P(E\cup F)}=\frac{P(E\cap F)}{P(E\cup F)}$. Then, $P(E\cap F|E\cup F)=\frac{P(E)P(F)}{P(E)P(F)-P(E)P(F)}=\frac{p_1p_2}{p_1+p_2-p_1p_2}$.
    \part The probability of Barbara hitting given that that at least one hit can also be found with the same formula. Now, we are looking for $P(E|E\cup F)$. Using the formula, we have $P(E|E\cup F)=\frac{P(E\cap (E\cup F))}{P(E\cup F)}=\frac{P(E)}{P(E\cup F)}$. Then, $P(E|E\cup F)=\frac{p_1}{p_1+p_2-p_1p_2}$.
    \part I assumed that the shots were independent, so that $P(E\cap F)=P(E)P(F)$ and $P(E\cup F)=P(E)+P(F)-P(E)P(F)$.
\end{parts}

\end{document}
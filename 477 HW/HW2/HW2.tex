\documentclass{exam}
\usepackage{amsmath}

\pagestyle{headandfoot}
\firstpageheadrule
\runningheadrule
\firstpageheader{Prof. Pham \\ Mathematical Theory of Probability}{Homework 2}{Aadhithya Saravanan}
\runningheader{Mathematical Theory of Probability \\ Homework 2}{}{Saravanan}
\firstpagefooter{}{}{}
\runningfooter{ }{\thepage}{ }

\printanswers

\begin{document}

\underline{Problem 1}
\newline
\begin{parts}
    \part We have 3 choices of subjects to pick 2 books from. Choosing 2 books from 6 math books gives \(\binom{6}{2}\) ways. Doing the same for the other books and summing gives a total of \(\binom{6}{2} + \binom{7}{2} + \binom{4}{2}\) ways.
    \part If the books are to be on different subjects, then the number of ways to choose one book from two different subjects is equal to the total number of ways to pick two books minus the number of ways to pick two books on the same subject. Thus, the number of ways is \(\binom{6+7+4}{2} - \left( \binom{6}{2} + \binom{7}{2} + \binom{4}{2} \right)\).
    \newline
\end{parts}

\underline{Problem 2}
\newline
\begin{parts}
    \part If two of the friends will not attend together, either 0 or 1 of them can attend. If 0 attend, then there are \(\binom{8-2}{5}\) ways to choose the 5 invited friends. If 1 attends, then there are \(\binom{8-2}{4}\) ways to choose 4 friends from the 6 friends that are not feuding and \(\binom{2}{1}\) ways to choose one of the feuding friends, giving \(\binom{8-2}{4}\binom{2}{1}\) ways. Thus, the total number of ways is \(\binom{8-2}{5} + \binom{8-2}{4}\binom{2}{1}\).
    \part If two of the friends will only attend together, either both or neither can attend. If neither attend, then there are \(\binom{8-2}{5}\) ways to choose the 5 invited friends. If both attend, then there are \(\binom{8-2}{5-2}\) ways to choose 3 more friends from the remaining 6 friends. Thus, the total number of ways is \(\binom{8-2}{5} + \binom{8-2}{3}\).
    \newline
\end{parts}

\underline{Problem 3}
\newline
\newline
A total of 4 steps have to be taken to the right and 3 steps have to be taken up. The total number of ways to get from A to B is the number of ways to arrange the sequence of steps. This is equal to choosing 4 items from 7, which is \(\binom{7}{4}\) ways.
\newline

\underline{Problem 4}
\newline
\newline
Each set of twins can be treated as one item, so we first look for the number of ways to order 3 items, which is \(3!\) ways. However, since each pair of twins can be arranged in 2 ways, we have to multiply by \(2^3\) to get a total of \(3! \cdot 2^3\) ways.
\newline

\underline{Problem 5}
\newline
\begin{parts}
    \part We are looking to arrange 10 items of 4 different types, of which items of the same type are indistinguishable. Thus, the number of ways to arrange the items is given by \(\binom{10}{3,4,2,1}\).
    \part If the United States has 1 competitor in the top three and 2 in the bottom three, we can multiply the number of ways that there can be 1 competitor in the top three, 2 in the bottom three, and the number of ways to arrange the remaining competitors. The number of ways to choose 1 competitor from the United States for the top three is \(\binom{3}{1}\). The number of ways to choose 2 competitors from the US for the bottom three is \(\binom{3}{2}\). The number of ways to arrange the remaining 7 competitors is \(\binom{7}{4,2,1}\). Thus, the total number of ways is \(\binom{3}{1}\binom{3}{2}\binom{7}{4,2,1}\).
    \newline
\end{parts}

\underline{Problem 6}
\newline
\newline
If the French and English delegates are to be seated next to each other, we can treat them as one item. Thus, we have 9 items to arrange, which can be done in \(9!\) ways. However, since the French and English delegates can be arranged in 2 ways, we have to multiply by 2 to get a total of \(9! \cdot 2\) ways. We then subtract the number of ways that the Russian and U.S. delegates can be seated next to each other by treating them as one item, giving us a total of 8 items to arrange. This can be done in \(8!\) ways, and since the two pairs can each be rearranged, we multiply by \(2^2\). Thus, the total number of ways is \(9! \cdot 2 - 8! \cdot 2^2\).
\newline
\end{document}